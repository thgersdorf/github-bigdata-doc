\documentclass[a4paper,10pt]{article}
\usepackage[utf8]{inputenc}
\usepackage[backend=bibtex]{biblatex}
\usepackage{filecontents}
\usepackage{graphics}
\usepackage[usenames,dvipsnames]{xcolor}
\usepackage{amsmath,amssymb,amsfonts,mathrsfs}
\usepackage{cancel}
\usepackage{mathtools}
\usepackage{ifthen}
\usepackage{minibox}
\usepackage{color}
\usepackage{bigdelim, bigstrut}
\usepackage{tocloft}
\usepackage{framed}
\usepackage{enumitem}

\usepackage[left=20mm,right=20mm,top=20mm,bottom=20mm,foot=10mm]{geometry}


%%%%%%%%%%%%%%%%%%%%%%%%%%%%%%%%%%%%%%%%%%%%%%%%%%%%%%%%%%%%%%%%%%%%%%%%%%%%%%%%%%%%%
%% CODE LISTINGS for source code of different programming languages in color
\usepackage{listings}
\usepackage{color}
\usepackage{textcomp}
\lstdefinelanguage{JavaScript}{
  keywords={typeof, new, true, false, catch, function, return, null, catch, switch, var, if, in, while, do, else, case, break},
  keywordstyle=\color{blue}\bfseries,
  ndkeywords={class, export, boolean, throw, implements, import, this},
  ndkeywordstyle=\color{darkgray}\bfseries,
  identifierstyle=\color{black},
  sensitive=false,
  comment=[l]{//},
  morecomment=[s]{/*}{*/},
  commentstyle=\color{purple}\ttfamily,
  stringstyle=\color{red}\ttfamily,
  morestring=[b]',
  morestring=[b]"
}


\definecolor{listinggray}{gray}{0.9}
\definecolor{lbcolor}{rgb}{0.9,0.9,0.9}
\lstset{
	backgroundcolor=\color{lbcolor},
	tabsize=4,
	rulecolor=,
	language=JavaScript,
	numbers=left,
	numberstyle=\footnotesize,
        basicstyle=\scriptsize,
        upquote=true,
        aboveskip={0.3\baselineskip},
        columns=fixed,
        showstringspaces=false,
        extendedchars=true,
        breaklines=true,
        prebreak = \raisebox{0ex}[0ex][0ex]{\ensuremath{\hookleftarrow}},
        frame=single,
        showtabs=false,
        showspaces=false,
        showstringspaces=false,
	basicstyle=\small\ttfamily,
        identifierstyle=\ttfamily,
%         keywordstyle=\color[rgb]{0,0,1},
        commentstyle=\color[rgb]{0.133,0.545,0.133},
        stringstyle=\color[rgb]{0.627,0.126,0.941},
}
%% end code listings
%%%%%%%%%%%%%%%%%%%%%%%%%%%%%%%%%%%%%%%%%%%%%%%%%%%%%%%%%%%%%%%%%%%%%%%%%%%%%%%%%%%%%









%opening
\title{Github Events Data}
\author{Thomas Gersdorf}

\addbibresource{smi.bib}

\setlength\parindent{0cm}

 

\input{../../math_setup/thg_math_setup.tex} % include math macros


\begin{document}

\maketitle


\section{Fundamental and Derived Measures}
\subsection{Terminology}
\begin{description}
\item[Issue] An issue is a generalized form of everything which should be improved on a piece of software, like for instance bug reports or new feature requests.
\item [Commit] A commit is the action of submitting new code to the repository. Typically, commits are made \textbf{decentralized}, which means that the developer is typing "commit" on his own computer. One or several commits can be uploaded via a Push.
\item [Push ] A push is a the action of uploading committed source code to the repository. It's one of the key features of Git's decentralized environment. During a push event, editing conflicts can emerge and then need to be resolved.
 \item[Gist] Gists are snippets of code to be shared with others, interdependent of a repository. Can be cloned, forked etc. like a repository.
\item [Fork] Forks are clones/copies of a repository or Gist into a User's own account. The can be used to parallelize development of the project, the feedback to the original project can be done via \textit{PullRequests.}
\item [PullRequest] 
\item [Payload] Typically describes the contents of an action (like the source code itself, the comments itself, all acompanying information but not the meta-data.)
\end{description}

Detailed information on Github's Event API can be found in \url{http://developer.github.com/v3/activity/events/types/} .

\subsection{Users}
\begin{center}
 
\begin{tabular}{|p{2.5cm}|p{5cm}|p{6cm}|p{2cm}|}\hline
\textbf{Event type} & \textbf{Description}&  \textbf{Contents}& \textbf{\#  for all users in example dataset} \\\hline
 IssuesEvent  & An issues is opened or closed.   &  Opened \& Closed Issues, Assignee (API) &29368\\\hline
CreateEvent    &    & &  116683\\\hline
PushEvent  & Source code is pushed into repository. &  Size, SHA Hash &  298729\\\hline
GistEvent   &    &   &  26267\\\hline
Issue
CommentEvent &    &    &   45166\\\hline
Commit
CommentEvent &    &    &  7827\\\hline
ForkEvent   &    &    & 19867\\\hline
WatchEvent  &  &    &   65193\\\hline
PullRequestEvent     &    &  &  20818\\\hline
PullRequest
ReviewCommentEvent  & && 3726\\\hline
FollowEvent   &    &  & 17049\\\hline
GollumEvent  &  Update of Wikipages, such as release notes& Page name (e.g. "Release Notes") , Action (created, edited), URL  &  11781\\\hline
DeleteEvent   & Either a branch or tag is removed.   &  Delete Type (Branch or Tag), Name of Deleted Object&   3785\\\hline
MemberEvent   & Users can be made collaborators, thereby given advanced "write" permissions to the repository.   & &    2764\\\hline
DownloadEvent  &    & & 2576\\\hline
PublicEvent &  &    &  553\\\hline
TotalEvents & Counts all of the above. & - & 672152 \\\hline
\end{tabular}
\end{center}




\subsubsection{Constructed measures}
\begin{description}
 \item [Issues\_Opened] \# of issues the user has opened
 \item [Issues\_Closed] \# of issues the user has closed
\item [Event\_Continuity] $\to$ TODO
\item [Events\_Own] Events on user's repository
\item [Events\_Active\_Involvement] Measure how much the user engages in projects of other user or organizations, events on other repositories 
\item [Events\_Passive\_Involvment] Measures how much other users engage in projects of the user, events on the user's repository by other users
\end{description}



\subsection{Repositories}

\subsubsection{Constructed measures}
\begin{description}
 \item [Distinct\_Contributing\_Users] Number of different users contributing via IssuesEvent, CreateEvent, PushEvent, PullRequestEvent
\end{description}


\section{Constructed Leadership Measures}
\subsection{Envisioning and Goal Orientation}

\begin{align}
 LS_{\text{milestone}} = \frac{\text{\# issue directly assigned to a milestone or release}}{\text{\# total issues opened}} 
\end{align}




\subsection{Leading People}
\subsubsection{Open Issue and Assign  User/Milestone}

\textbf{Measure:}
\begin{align}
 LS_{\text{assign}} = \frac{\text{\# issue directly assigned to other person}}{\text{\# total issues opened}} 
\end{align}


\textbf{Reasoning:}
The more people directly assign users to issues, the more they know about the people's capabilities and empower other users to solve this issue.



\textbf{Control for:}
\begin{itemize}
 \item Total number of issues
\item Size of project
\end{itemize}



Assignee currently only via API from \texttt{ https://api.github.com/repos/thgersdorf/github-play/issues/1
}, see reference.



\subsection{Empowering}


\subsection{Project Governance}


\subsubsection{Closing issues by other people's commits | Gatekeepers}
\textbf{Measure:}
\begin{align}
 LS_{\text{assign}} = \frac{\text{\# issue directly assigned to other person}}{\text{\# total issues opened}} 
\end{align}

\textbf{Control for:}




\section{Performance Measures}
\subsection{Contributors}
\subsection{Release, Commits, Issue closures}

\begin{align}
 \text{Issue closing} = \frac{}{}
\end{align}


\textbf{Issue User Effectiveness:} 
\begin{align}
 IUE = \frac{\Delta \frac{\text{\# Issues Open }}{\text{\# Total Issues}}}{\Delta\text{\# User contributing}}
\end{align}



\subsection{Downloads, Views etc.}


\section{Reference}
\subsection{Assignee from API}
Example URL: \texttt{https://api.github.com/repos/thgersdorf/github-play/issues/1}

\begin{lstlisting}
 {
  "url": "https://api.github.com/repos/kevans91/Chatters/issues/5",
  "labels_url": "https://api.github.com/repos/kevans91/Chatters/issues/5/labels{/name}",
  "comments_url": "https://api.github.com/repos/kevans91/Chatters/issues/5/comments",
  "events_url": "https://api.github.com/repos/kevans91/Chatters/issues/5/events",
  "html_url": "https://github.com/kevans91/Chatters/issues/5",
  "id": 4059710,
  "number": 5,
  "title": "Locked Menu Entries",
  "user": {
    "login": "kevans91",
    "id": 646318,
    "avatar_url": "https://gravatar.com/avatar/8d4c2928fe43bb196970867f59ab6b93?d=https%3A%2F%2Fidenticons.github.com%2F5b22ea77797a2c6e6e0859b128be4d94.png&r=x",
    "gravatar_id": "8d4c2928fe43bb196970867f59ab6b93",
    "url": "https://api.github.com/users/kevans91",
    "html_url": "https://github.com/kevans91",
    "followers_url": "https://api.github.com/users/kevans91/followers",
    "following_url": "https://api.github.com/users/kevans91/following{/other_user}",
    "gists_url": "https://api.github.com/users/kevans91/gists{/gist_id}",
    "starred_url": "https://api.github.com/users/kevans91/starred{/owner}{/repo}",
    "subscriptions_url": "https://api.github.com/users/kevans91/subscriptions",
    "organizations_url": "https://api.github.com/users/kevans91/orgs",
    "repos_url": "https://api.github.com/users/kevans91/repos",
    "events_url": "https://api.github.com/users/kevans91/events{/privacy}",
    "received_events_url": "https://api.github.com/users/kevans91/received_events",
    "type": "User",
    "site_admin": false
  },
  "labels": [
    {
      "url": "https://api.github.com/repos/kevans91/Chatters/labels/enhancement",
      "name": "enhancement",
      "color": "84b6eb"
    }
  ],
  "state": "open",
  "assignee": null,
  "milestone": null,
  "comments": 0,
  "created_at": "2012-04-11T06:43:08Z",
  "updated_at": "2012-04-11T06:43:08Z",
  "closed_at": null,
  "pull_request": {
    "html_url": null,
    "diff_url": null,
    "patch_url": null
  },
  "body": "Locked menu entries currently clutter up the menus at the top pretty badly. They either need to be removed or used.",
  "closed_by": null
}
\end{lstlisting}

\subsection{Raw JSONs}
\subsubsection{IssuesEvent}

\begin{lstlisting}
 "repository":{"homepage":"http://tdurand.github.com/mapafortaleza",
"has_downloads":true,"has_issues":true,"master_branch":"gh-pages","forks":5,"language":"JavaScript","fork":false,"has_wiki":true,"url":"https://github.com/tdurand/mapafortaleza","created_at":"2011/09/29 03:27:42 -0700","size":132,"private":false,"name":"mapafortaleza","description":"Mapa de Fortaleza","owner":"tdurand","open_issues":9,"watchers":13,"pushed_at":"2012/02/29 16:43:30 -0800"},
"actor_attributes":{"name":"Thibault Durand","company":"@tibbb","gravatar_id":"cedb88f236afaeed4caea39917ebd0a7","blog":"http://www.thibault-durand.fr","type":"User","login":"tdurand"}
,"created_at":"2012/04/11 16:01:27 -0700","public":true,"actor":"tdurand","payload":{"number":12,"action":"opened","issue":4075341},"url":"https://github.com/tdurand/mapafortaleza/issues/12","type":"
IssuesEvent"}
\end{lstlisting}

\subsubsection{PushEvent}

\begin{lstlisting}
{"repository":{"homepage":"","has_downloads":true,"has_issues":true,"
forks":1,"fork":false,"has_wiki":true,"url":"https://github.com/brunolopesjn/JConvexHull","created_at":"2012/04/10 18:37:34 -0700","size":208,"private":false,"name":"JConvexHull","description":"Programa Java Swing que resolve o problema do caixeiro viajante (TSP) utilizando o Convexo de Hull","owner":"brunolopesjn","open_issues":0,"watchers":1,"pushed_at":"2012/04/11 15:01:36 -0700"},"actor_attributes":{"name":"Bruno Lopes Alcantara Batista","gravatar_id":"854bf813e2a302dcc4a059b89c502424","blog":"kariridev.blogspot.com","type":"User","login":"brunolopesjn","email":"brunolopesjn@gmail.com"},"created_at":"2012/04/11 15:01:37 -0700","public":true,"actor":"brunolopesjn","payload":{"head":"8469f67ff919f2a776336bfe61f5ae7f40e76c72","size":1,"shas":[["8469f67ff919f2a776336bfe61f5ae7f40e76c72","brunolopesjn@gmail.com","Criado a interface grafica base, de desenho dos pontos, o desenho\ndo plano cartesiano e a ica de abertura de arquivos.","Bruno Lopes Alcantara Batista",true]],"ref":"refs/heads/master"},"
url":"https://github.com/brunolopesjn/JConvexHull/compare/ea77edbd70...8469f67ff9","type":"PushEvent"}
\end{lstlisting}

\subsubsection{IssueCommentEvent}
\begin{lstlisting}

\end{lstlisting}

\subsubsection{CommitCommentEvent}
\begin{lstlisting}

\end{lstlisting}

\subsubsection{ForkEvent}
\begin{lstlisting}

\end{lstlisting}

\subsubsection{FollowEvent}
\begin{lstlisting}

\end{lstlisting}


\subsubsection{GollumEvent}
\begin{lstlisting}
{"repository":{"homepage":"http://www.sisodb.com","has_downloads":true,"has_issues":true,"forks":4,"language":"C#","fork":false,"has_wiki":true,"url":"https://github.com/
danielwertheim/SisoDb-Provider","created_at":"2011/01/31 05:33:16 -0800","size":3472,"private":false,"name":"SisoDb-Provider","description":"SisoDb - Simple Structure Oriented Db","owner":"danielwertheim","open_issues":0,"watchers":52,"pushed_at":"2012/04/11 14:52:42 -0700"},"actor_attributes":{"name":"Daniel Wertheim","gravatar_id":"274ce0291206c9d27635a865ac48b5c4","location":"Sweden","blog":"http://daniel.wertheim.se","type":"User","login":"danielwertheim"},"created_at":"2012/04/11 15:01:37 -0700","public":true,"actor":"danielwertheim","payload":{"pages":[{"sha":"909e1049863468ad462655505441bf7bf39dc739","title":"Release notes","action":"edited","page_name":"Release notes","summary":null,"html_url":"https://github.com/danielwertheim/SisoDb-Provider/wiki/Release-notes"}]},"url":"https://github.com/danielwertheim/SisoDb-Provider/wiki/Release-notes","type":"GollumEvent"}
\end{lstlisting}

\subsubsection{DeleteEvent}
\begin{lstlisting}
{"repository":{"homepage":"","has_downloads":true,"has_issues":false,"master_branch":"master","forks":0,"language":"Shell","fork":true,"has_wiki":true,"url":"https://github.com/travp624/android_device_moto_shadow","created_at":"2012/04/03 01:32:24 -0700","size":6060,"private":false,"name":"android_device_moto_shadow","description":"","owner":"travp624","open_issues":0,"watchers":1,"pushed_at":"2012/04/11 15:02:07 -0700"},"actor_attributes":{"name":"teamicemods","gravatar_id":"1eb4d77c7c79eccde05a7047fbae4195","location":"wisconsin home of cheese baby","blog":"","type":"User","login":"travp624","email":"travp624@gmail.com"},"created_at":"2012/04/11 15:02:07 -0700","public":true,"actor":"travp624","payload":{"ref_type":"branch","ref":"ics"},"url":"https://github.com/","type":"DeleteEvent"}
\end{lstlisting}


\subsection{SQL Queries Temp}
\lstset{
	backgroundcolor=\color{lbcolor},
	tabsize=4,
	rulecolor=,
	language=SQL,
	numbers=left,
	numberstyle=\footnotesize,
        basicstyle=\scriptsize,
        upquote=true,
        aboveskip={0.3\baselineskip},
        columns=fixed,
        showstringspaces=false,
        extendedchars=true,
        breaklines=true,
        prebreak = \raisebox{0ex}[0ex][0ex]{\ensuremath{\hookleftarrow}},
        frame=single,
        showtabs=false,
        showspaces=false,
        showstringspaces=false,
	basicstyle=\small\ttfamily,
        identifierstyle=\ttfamily,
%         keywordstyle=\color[rgb]{0,0,1},
        commentstyle=\color[rgb]{0.133,0.545,0.133},
        stringstyle=\color[rgb]{0.627,0.126,0.941},
}


\subsubsection{Generic Event Queries}
\begin{lstlisting}
 # total events involvements:
 SELECT * FROM `events` WHERE `actor` = 'mitar' AND `repository_owner` != 'mitar'
SELECT * FROM `events` WHERE `actor` = 'mitar' AND `repository_owner` = 'mitar'

 # specific Issues properties
 SELECT payload_number, payload_action, payload_issue from events where type   = 'IssuesEvent' LIMIT 30;


# user profiling
INSERT INTO eng_users (user, TotalEvents,  IssuesEvent , CreateEvent , PushEvent , GistEvent , IssueCommentEvent , CommitCommentEvent , ForkEvent , WatchEvent , PullRequestEvent , PullRequestReviewCommentEvent , FollowEvent , GollumEvent , DeleteEvent , MemberEvent , DownloadEvent , PublicEvent, Issues_Opened, Issues_Closed, Events_Own, Events_Active_Involvement, Events_Passive_Involvment)  SELECT '" + user_i[0] +"' , COUNT(*),  SUM(Type = 'IssuesEvent'),  SUM(Type = 'CreateEvent'), SUM(Type = 'PushEvent'), SUM(Type = 'GistEvent'), SUM(Type = 'IssueCommentEvent'),SUM(Type = 'CommitCommentEvent'),SUM(Type = 'ForkEvent'),SUM(Type = 'WatchEvent'),SUM(Type = 'PullRequestEvent'),SUM(Type = 'PullRequestReviewCommentEvent'),SUM(Type = 'FollowEvent'),SUM(Type = 'GollumEvent'),   SUM(Type = 'DeleteEvent'), SUM(Type = 'MemberEvent'), SUM(Type = 'DownloadEvent'), SUM(Type = 'PublicEvent'), SUM(type = 'IssuesEvent' AND payload_action='opened'), SUM(type = 'IssuesEvent' AND payload_action='closed'),   SUM(actor = '" + 
user_i[0] +"' AND `repository_owner` = '" + user_i[0] +"'),  SUM(actor = '" + user_i[0] +"' AND `repository_owner` != '" + user_i[0] +"') , SUM( actor != '" + user_i[0] +"' AND `repository_owner` = '" + user_i[0] +"'   )   FROM events WHERE actor_attributes_type = 'User' AND actor_attributes_login = '" + user_i[0] +"' ON DUPLICATE KEY UPDATE user = '" + user_i[0] +"' ;
# passive involvement:
UPDATE eng_users SET Events_Passive_Involvment = ( SELECT COUNT( * )  FROM  `events`  WHERE  `repository_organization` IS NULL  AND   `repository_owner` =  'torvalds' AND actor !=  'torvalds' )  WHERE eng_users.user =  'torvalds'
\end{lstlisting}


\section{Performance Optimization}
Lessons learned:
\begin{itemize}
 \item 
\end{itemize}



\end{document}

